\documentclass{beamer}

\mode<presentation> {
\usetheme{Madrid}
}
\usepackage{graphicx} % Allows including images
\usepackage{booktabs} % Allows the use of \toprule, \midrule and \bottomrule in tables
\usepackage{listings}
\lstset
{
    language=Ruby,
    breaklines=true,
    basicstyle=\tt\scriptsize,
    keywordstyle=\color{blue},
    identifierstyle=\color{magenta},
}

%----------------------------------------------------------------------------------------
% TITLE PAGE
%----------------------------------------------------------------------------------------

\title[Ruby]{El lenguajge Ruby}

\author{Ariel Herrera \\ Sa\'ul Zamora}
\institute[ITCR]
{
Instituto Tecnol\'ogico de Costa Rica \\ Escuela de Ingenier\'ia en Computaci\'on \\ Lenguajes de Programaci\'on
\medskip
}

\begin{document}

\begin{frame}
\titlepage
\end{frame}

\begin{frame}
\frametitle{\'Indice}
\tableofcontents
\end{frame}

%----------------------------------------------------------------------------------------
% PRESENTATION SLIDES
%----------------------------------------------------------------------------------------

%------------------------------------------------
\section{Datos hist\'oricos}
\section{Importancia y usos}
\section{Tipos de datos}
\section{Expresiones}
\section{Estructuras de control}
\section{Caracter\'isticas}
\section{Ventajas y desventajas}
\section{Demo}

\begin{frame}
\frametitle{Datos hist\'oricos}
Ruby es un lenguaje de prop\'osito general, din\'amico, reflectivo y orientado a objetos. Fu\'e dise\~nado y desarrollado a mediados de los a\~nos 90 por Yukihiro "Matz" Matsumoto en Jap\'on.

De acuerdo con su creador, Ruby fue influenciado por Perl, Smalltalk, Eiffel, Ada y LISP. Soporta la programaci\'on en m\'ultiples paradigmas incluyendo funcional, orientaci\'on a objetos e imperativo. Tambi\'en tiene un sistema de tipos din\'amicos y un manejo de memoria autom\'atico.
\end{frame}

%------------------------------------------------

\begin{frame}
\frametitle{Importancia y usos}
Ruby es un lenguaje f\'acil de aprender, con fuertes abstracciones de los detalles computacioneles; dado esto es un buen lenguaje a considerar para comenzar a aprender a programar.

Ruby tambi\'en es la entrada a Ruby On Rails. RoR es un framework con mucha popularidad que usa y depende de Ruby, dicho framework es ampliamente utilizado en la creaci\'on de aplicaciones web.
\end{frame}

%------------------------------------------------

\begin{frame}
\frametitle{Tipos de datos}
Ruby posee un tipado de datos din\'amico, sin embargo posee los siguientes tipos de datos nativos:
\begin{itemize}
  \item String
  \item Fixnum
  \item Integer
  \item Numeric
  \item Float
  \item NilClass
  \item Hash
  \item Symbol
  \item Array
  \item Range
\end{itemize}
\end{frame}

%------------------------------------------------

\begin{frame}
\frametitle{Expresiones}
\begin{columns}[c] % The "c" option specifies centered vertical alignment while the "t" option is used for top vertical alignment

\column{.5\textwidth} % Left column and width
\begin{itemize}
  \item Operadores aritm\'eticos:
    \begin{itemize}
      \item +, -, *, /, \%, ** (potencia)
    \end{itemize}
\end{itemize}

\frametitle{}
\begin{itemize}
  \item Operadores de comparaci\'on:
  \begin{itemize}
    \item ==, !=, >, <, >=, <=, <=>, ===, .eql?, equal?
  \end{itemize}
\end{itemize}

\column{.5\textwidth} % Right column and width
\begin{itemize}
  \item Operadores de asignaci\'on:
  \begin{itemize}
    \item =, +=, -=, *=, /=, \%=, **=
  \end{itemize}
\end{itemize}

\begin{itemize}
  \item Operadores l\'ogicos:
  \begin{itemize}
    \item and, or, \&\&, !, not
  \end{itemize}
\end{itemize}
\end{columns}
\end{frame}

\begin{frame}[fragile]
\frametitle{Estructuras de control}
\begin{columns}[c] % The "c" option specifies centered vertical alignment while the "t" option is used for top vertical alignment

\column{.5\textwidth} % Left column and width
Estatuto IF:
\begin{lstlisting}
# Generate a random number and print whether it's even or odd.
if rand(100) % 2 == 0
  puts "It's even"
else
  puts "It's odd"
end

puts "It's even" if rand(100) % 2 == 0
\end{lstlisting}

Estatuto UNLESS:
\begin{lstlisting}
# Generate a random number and print whether it's even or odd.
unless rand(100) % 2 == 0
  puts "It's even"
else
  puts "It's odd"
end
\end{lstlisting}

\frametitle{}
\column{.5\textwidth} % Right column and width
Bloques e iteradores:
\begin{lstlisting}
{ puts 'Hello, World!' } # note the braces
# or:
do
  puts 'Hello, World!'
end

File.open('file.txt', 'w') do |file| # 'w' denotes "write mode"
  file.puts 'Wrote some text.'
end                                  # file is automatically closed here

File.readlines('file.txt').each do |line|
  puts line
end
# => Wrote some text.
\end{lstlisting}
\end{columns}
\end{frame}

%------------------------------------------------

\begin{frame}
\frametitle{Caracter\'isticas principales}
\begin{itemize}
  \item Fuerte orientaci\'on a objetos con herencia, mixins y metaclases.
  \item Tipado de datos din\'amico y \emph{duck typing} (si se ve como un pato y suena como un pato, es un pato).
  \item Todo es una expresi\'on (hasta los estatutos) y todo se ejecuta imperativamente (hasta las declaraciones).
  \item Reflexi\'on y alteraci\'on din\'amica de objetos para facilitar metaprogramaci\'on.
  \item Sint\'axis \'unica de bloques para iteradores y generadores.
  \item Notaci\'on literal para arreglos, hashes, expresiones regulares y s\'imbolos.
  \item Interpolaci\'on de hileras.
  \item Argumentos default.
  \item Recolector de basura.
\end{itemize}
\end{frame}

%------------------------------------------------

\begin{frame}
\frametitle{Caracter\'isticas distintivas}
\begin{itemize}
  \item Cuatro niveles de alcance para variables: globales, de clase, de instancia y local.
  \item Sobrecarga de operadores.
  \item Soporte nativo para n\'umeros racionales, complejos y aritm\'etica de presici\'on arbitraria.
  \item Soporte inicial para Unicode y m\'ultiples codificaciones de caract\'eres.
  \item Consola Ruby interactiva.
  \item Manejo centralizado de paquetes a trav\'ez de \emph{RugyGems}.
  \item Implementaciones en todas las plataformas m\'as conocidas.
\end{itemize}
\end{frame}

%------------------------------------------------

\begin{frame}
\frametitle{Ventajas}
\begin{itemize}
  \item Herramientas: utilizando Ruby On Rails, es posible entregar m\'as \emph{features} en menos tiempo. El framework provee de una estructura est\'andar para web apps.
  \item Disponibilidad de ibrer\'ias.
  \item Comunidad grande y activa.
  \item Productividad: Ruby es un lenguaje elocuente, lo que combinado con la gran variedad de librer\'ias externas, habilita al programador al desarrollo r\'apido.
\end{itemize}
\end{frame}

%------------------------------------------------

\begin{frame}
\frametitle{Desventajas}
\begin{itemize}
  \item Velocidad de ejecuci\'on: el mayor argumento contra Ruby es que es lento. Sin embargo, estos problemas de velocidad no se notar\'an hasta que la aplicaci\'on tenga un gran volumen de tr\'afico.
  \item Velocidad de arranque: aplicable en el caso del framework, Ruby on Rails, dependiendo del n\'umero de dependencias (\emph{gemas}), puede tomar una cantidad de tiempo significativa a una aplicaci\'on para arrancar, lo que resiente el tiempo de desarrollo.
  \item Documentaci\'on: puede ser dif\'icil encontrar buena documentaci\'on, especialmente para las gemas y librer\'ias que hacen uso excesivo de mixins.
\end{itemize}
\end{frame}

%------------------------------------------------

\begin{frame}
\Huge{\centerline{Demostraci\'on}}
\end{frame}

%------------------------------------------------

\begin{frame}
\Huge{\centerline{Gracias por su atenci\'on}}
\end{frame}

%----------------------------------------------------------------------------------------

\end{document} 