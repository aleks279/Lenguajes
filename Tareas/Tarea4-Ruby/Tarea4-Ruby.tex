\documentclass{IEEEtran}
\usepackage{graphicx}
\usepackage{fancyhdr}
\usepackage{listings}
\usepackage{xcolor}
\usepackage{float}
\lstset
{
    language=Ruby,
    breaklines=true,
    basicstyle=\tt\scriptsize,
    keywordstyle=\color{blue},
    identifierstyle=\color{magenta},
}

\graphicspath{ {images/} }
\pagestyle{fancy}
\fancyhf{}
\rhead{Tarea 4 - Ruby}
\rfoot{P\'agina \thepage}

\begin{document}
\begin{titlepage}
  \centering
  {\scshape\LARGE Instituto Tecnol\'ogico de Costa Rica \par}
  \vspace{1cm}
  {\scshape\Large Tarea 4 - Ruby\par}
  \vspace{1.5cm}
  {\Large\itshape Ariel Herrera\par}
  {\Large\itshape Sa\'ul Zamora\par}
  \vfill
  profesor\par
  M. Sc. Sa\'ul Calder\'on Ram\'irez \textsc{}

  \vfill

% Bottom of the page
  {\large \today\par}
\end{titlepage}

\section{Datos hist\'oricos}
Ruby es un lenguaje de prop\'osito general, din\'amico, reflectivo y orientado a objetos. Fu\'e dise\~nado y desarrollado a mediados de los a\~nos 90 por Yukihiro "Matz" Matsumoto en Jap\'on.

De acuerdo con su creador, Ruby fue influenciado por Perl, Smalltalk, Eiffel, Ada y LISP. Soporta la programaci\'on en m\'ultiples paradigmas incluyendo funcional, orientaci\'on a objetos e imperativo. Tambi\'en tiene un sistema de tipos din\'amicos y un manejo de memoria autom\'atico.

\section{Importancia y usos}
Ruby es un lenguaje f\'acil de aprender, con fuertes abstracciones de los detalles computacioneles; dado esto es un buen lenguaje a considerar para comenzar a aprender a programar.

Ruby tambi\'en es la entrada a Ruby On Rails. RoR es un framework con mucha popularidad que usa y depende de Ruby, dicho framework es ampliamente utilizado en la creaci\'on de aplicaciones web.

\section{Tipos de datos}
Ruby posee un tipado de datos din\'amico, sin embargo posee los siguientes tipos de datos nativos:
\begin{itemize}
  \item String
  \item Fixnum
  \item Integer
  \item Numeric
  \item Float
  \item NilClass
  \item Hash
  \item Symbol
  \item Array
  \item Range
\end{itemize}

\section{Expresiones}
\subsection{Operadores aritm\'eticos}
+, -, *, /, \%, ** (potencia)

\subsection{Operadores de comparaci\'on}
==, !=, >, <, >=, <=, <=>, ===, .eql?, equal?

\subsection{Operadores de asignaci\'on}
=, +=, -=, *=, /=, \%=, **=

\subsection{Operadores l\'ogicos}
and, or, \&\&, !, not

\section{Estructuras de control}
Estatuto IF:
\begin{lstlisting}
# Generate a random number and print whether it's even or odd.
if rand(100) % 2 == 0
  puts "It's even"
else
  puts "It's odd"
end
\end{lstlisting}

Bloques e iteradores:
\begin{lstlisting}
{ puts 'Hello, World!' } # note the braces
# or:
do
  puts 'Hello, World!'
end

File.open('file.txt', 'w') do |file| # 'w' denotes "write mode"
  file.puts 'Wrote some text.'
end                                  # file is automatically closed here

File.readlines('file.txt').each do |line|
  puts line
end
# => Wrote some text.
\end{lstlisting}

\section{Caracter\'isticas principales}
\begin{itemize}
  \item Fuerte orientaci\'on a objetos con herencia, mixins y metaclases.
  \item Tipado de datos din\'amico y \emph{duck typing} (si se ve como un pato y suena como un pato, es un pato).
  \item Todo es una expresi\'on (hasta los estatutos) y todo se ejecuta imperativamente (hasta las declaraciones).
  \item Reflexi\'on y alteraci\'on din\'amica de objetos para facilitar metaprogramaci\'on.
  \item Sint\'axis \'unica de bloques para iteradores y generadores.
  \item Notaci\'on literal para arreglos, hashes, expresiones regulares y s\'imbolos.
  \item Interpolaci\'on de hileras.
  \item Argumentos default.
\end{itemize}

\section{Caracter\'isticas distintivas}
\begin{itemize}
  \item Recolector de basura.
  \item Cuatro niveles de alcance para variables: globales, de clase, de instancia y local.
  \item Sobrecarga de operadores.
  \item Soporte nativo para n\'umeros racionales, complejos y aritm\'etica de presici\'on arbitraria.
  \item Soporte inicial para Unicode y m\'ultiples codificaciones de caract\'eres.
  \item Consola Ruby interactiva.
  \item Manejo centralizado de paquetes a trav\'ez de \emph{RugyGems}.
  \item Implementaciones en todas las plataformas m\'as conocidas.
\end{itemize}

\section{Ventajas y desventajas}
\subsection{Ventajas}
\begin{itemize}
  \item Herramientas: utilizado con el framework de Ruby On Rails, Ruby es bueno para entregar m\'as \emph{features} en menos tiempo. Provee de una estructura est\'andar para web apps, donde todos los patrones esnt\'andar son manejados por el usuario.
  \item Librer\'ias: existen gemas (librer\'ias externas) para casi todo lo que se le pueda ocurrir al usuario.
  \item Calidad de c\'odigo: generalmente, la calidad de c\'odigo Ruby es mejor que la de sus contrapartes en PHP o NodeJS.
  \item Pruebas automatizadas: la comunidad Ruby soporta mucho las pruebas, y a\'un m\'as, las pruebas automatizadas. Lo cual ayuda a entregar software de buena calidad.
  \item Gran comunidad: en casi todas las grandes ciudades del mundo hay una comunidad Ruby que dirige reuniones regulares. Ruby es uno de los lenguajes m\'as populares en Github.
  \item Es popular en el valle: la historia a demostrado que las tecnolog\'ias que son populares en Silicon Valley son gradualmente aceptadas en todo el mundo. Muchas de las grandes startups de los a\~nos recientes como AirBnB, Etsy, Github y Shopify, est\'an construidas con Ruby.
  \item Productividad: Ruby es un lenguaje elocuente, lo que combinado con la gran variedad de librer\'ias externas, habilita al programador al desarrollo r\'apido.
\end{itemize}
\subsection{Desventajas}
\begin{itemize}
  \item Velocidad de ejecuci\'on: el mayor argumento contra Ruby es que es lento. Sin embargo, estos problemas de velocidad no se notar\'an hasta que la aplicaci\'on tenga un gran volumen de tr\'afico.
  \item Velocidad de arranque: aplicable en el caso del framework, Ruby on Rails, dependiendo del n\'umero de dependencias (\emph{gemas}), puede tomar una cantidad de tiempo significativa a una aplicaci\'on para arrancar, lo que resiente el tiempo de desarrollo.
  \item Documentaci\'on: puede ser dif\'icil encontrar buena documentaci\'on, especialmente para las gemas y librer\'ias que hacen uso excesivo de mixins.
  \item M\'ultiples hilos: Ruby on Rails soporta m\'ultiples hilos, aunque algunas de las librer\'ias de entrada y salida no. Esto significa que si no se es cuidadoso, los requests se apilaran detr\'as del request actual lo cual puede introducir problemas de performance.
  \item ActiveRecord: es una librer\'ia fuertemente utilizada en Ruby on Rails. Y si bien est\'a bien dise\~nada, el m\'as grande problema es que el dominio se amarra muy fuerte a los mecanismos de persisstencia.
\end{itemize}
\section{Ejemplo}
\subsection{Strings}
\begin{lstlisting}
a = "\nThis is a double-quoted string\n"
a = %Q{\nThis is a double-quoted string\n}
a = %{\nThis is a double-quoted string\n}
a = %/\nThis is a double-quoted string\n/
a = <<-BLOCK

This is a double-quoted string
BLOCK

var = 3.14159 # interpolacion de variables
"pi is #{var}"
=> "pi is 3.14159"
\end{lstlisting}

\subsection{Colecciones}
\begin{lstlisting}
a = [1, 'hi', 3.14, 1, 2, [4, 5]]

a[2]             # => 3.14
a.[](2)          # => 3.14
a.reverse        # => [[4, 5], 2, 1, 3.14, 'hi', 1]
a.flatten.uniq   # => [1, 'hi', 3.14, 2, 4, 5]
\end{lstlisting}

\section{Referencias}

\begin{thebibliography}{99}

\bibitem{wiki} Ruby (programming language) (2016). . In \emph{Wikipedia}. Retrieved from \texttt{https://en.wikipedia.org/wiki/Ruby\_(programming\_language)}
\bibitem{ruby} MacDonald, R., \& Founder. (2015, September 8). Pros and cons of Ruby on rails. Retrieved September 24, 2016, from \texttt{https://www.madetech.com/blog/pros-and-cons-of-ruby-on-rails}
\bibitem{datos} Ruby programming/data types - Wikibooks, open books for an open world. Retrieved September 24, 2016, from \texttt{https://en.wikibooks.org/wiki/Ruby\_Programming/Data\_types}
\bibitem{operadores} Ruby Operators. (2016). Retrieved September 24, 2016, from \texttt{https://www.tutorialspoint.com/ruby/ruby\_operators.htm}
\bibitem{importance} Retrieved September 25, 2016, from http://www.skilledup.com/articles/4-reasons-learn-ruby-first-programming-language

\end{thebibliography}

\end{document}
