\documentclass{beamer}

\mode<presentation> {
\usetheme{Madrid}
}
\usepackage{graphicx} % Allows including images
\usepackage{booktabs} % Allows the use of \toprule, \midrule and \bottomrule in tables
\usepackage{listings}
\lstset
{
    language=LISP,
    breaklines=true,
    basicstyle=\tt\scriptsize,
    keywordstyle=\color{blue},
    identifierstyle=\color{magenta},
}

%----------------------------------------------------------------------------------------
% TITLE PAGE
%----------------------------------------------------------------------------------------

\title[LISP]{El lenguajge LISP}

\author{Ariel Herrera \\ Sa\'ul Zamora}
\institute[ITCR]
{
Instituto Tecnol\'ogico de Costa Rica \\ Escuela de Ingenier\'ia en Computaci\'on \\ Lenguajes de Programaci\'on
\medskip
}

\begin{document}

\begin{frame}
\titlepage
\end{frame}

\begin{frame}
\frametitle{\'Indice}
\tableofcontents
\end{frame}

%----------------------------------------------------------------------------------------
% PRESENTATION SLIDES
%----------------------------------------------------------------------------------------

%------------------------------------------------
\section{Datos hist\'oricos}
\section{Importancia y usos}
\section{Tipos de datos}
\section{Expresiones}
\section{Estructuras de control}
\section{Caracter\'isticas}
\section{Ventajas y desventajas}
\section{Demo}

\begin{frame}
\frametitle{Datos hist\'oricos}
\begin{itemize}
  \item Dise\~nado en 1958 por John McCarthy.
  \item Segundo lenguaje de alto nivel de la historia.
  \item LISt Processor.
  \item Creado como lenguaje de notaci\'on matem\'atica para programas de computac\'on.
\end{itemize}
\end{frame}

%------------------------------------------------

\begin{frame}
\frametitle{Importancia y usos}
\begin{itemize}
  \item Da origen a dialectos como \emph{Scheme}, ampliamente usados en la educaci\'on.
  \item Ampliamente utilizado en la investigaci\'on de la inteligencia artificial.
\end{itemize}

\end{frame}

%------------------------------------------------

\begin{frame}
\frametitle{Tipos de datos}
En LISP, las variables no son tipadas, pero los objetos s\'i.
Los tipos de datos en LISP pueden ser categorizados como:
\begin{itemize}
  \item Escalares:
  \begin{itemize}
    \item tipos num\'ericos
    \item caract\'eres
    \item s\'imbolos
  \end{itemize}
  \item Estructuras de datos:
  \begin{itemize}
    \item listas
    \item vectores
    \item vectores de bits (\emph{bit-vectors})
    \item strings
  \end{itemize}
\end{itemize}
\end{frame}

%------------------------------------------------

\begin{frame}
\frametitle{Expresiones}
\begin{columns}[c] % The "c" option specifies centered vertical alignment while the "t" option is used for top vertical alignment

\column{.5\textwidth} % Left column and width
\begin{itemize}
  \item Operadores aritm\'eticos:
    \begin{itemize}
      \item +, -, *, /, quotient, gcd (greater common divisor), lcm (least common multiple)
    \end{itemize}
\end{itemize}

\column{.5\textwidth} % Right column and width

\begin{itemize}
  \item Operadores l\'ogicos:
  \begin{itemize}
    \item and, nor, not, xor, or
  \end{itemize}
\end{itemize}
\end{columns}
\end{frame}

\begin{frame}
\frametitle{Estructuras de control}
Originalmente, LISP tiene pocas estructuras de control:
\begin{itemize}
  \item cond
\end{itemize}

Dialectos como Scheme:
\begin{itemize}
  \item do
  \item if
  \item let
\end{itemize}
\end{frame}

%------------------------------------------------

\begin{frame}
\frametitle{Caracter\'isticas}
Principales:
\begin{itemize}
  \item Extensivo uso de par\'entesis.
  \item Influencia del c\'alculo lambda.
  \item Todo son funciones.
\end{itemize}

Distintivas:
\begin{itemize}
  \item Introducci\'on del concepto de recolector de basura. 
\end{itemize}
\end{frame}

%------------------------------------------------

\begin{frame}
\frametitle{Pros \& Cons}
\begin{columns}[c] % The "c" option specifies centered vertical alignment while the "t" option is used for top vertical alignment

\column{.5\textwidth} % Left column and width
Ventajas:
\begin{itemize}
  \item Manejo nativo de listas.
  \item Gran poder de expresividad.
\end{itemize}

\column{.5\textwidth} % Right column and width
Desventajas:
\begin{itemize}
  \item Gran cantidad de par\'entesis.
\end{itemize}
\end{columns}
\end{frame}

\begin{frame}
\Huge{\centerline{Demostraci\'on}}
\end{frame}

%------------------------------------------------

\begin{frame}
\Huge{\centerline{Gracias por su atenci\'on}}
\end{frame}

%----------------------------------------------------------------------------------------

\end{document} 