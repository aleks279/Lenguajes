\documentclass{IEEEtran}
\usepackage{graphicx}
\usepackage{fancyhdr}
\usepackage{listings}
\usepackage{xcolor}
\usepackage{float}
\lstset
{
    language=Oz,
    breaklines=true,
    basicstyle=\tt\scriptsize,
    keywordstyle=\color{blue},
    identifierstyle=\color{magenta},
}

\graphicspath{ {images/} }
\pagestyle{fancy}
\fancyhf{}
\rhead{Tarea 3 - Oz (Mozart)}
\rfoot{P\'agina \thepage}

\begin{document}
\begin{titlepage}
  \centering
  {\scshape\LARGE Instituto Tecnol\'ogico de Costa Rica \par}
  \vspace{1cm}
  {\scshape\Large Tarea 3 - Oz (Mozart)\par}
  \vspace{1.5cm}
  {\Large\itshape Ariel Herrera\par}
  {\Large\itshape Sa\'ul Zamora\par}
  \vfill
  profesor\par
  M. Sc. Sa\'ul Calder\'on Ram\'irez \textsc{}

  \vfill

% Bottom of the page
  {\large \today\par}
\end{titlepage}

\section{Datos hist\'oricos}
Oz es un lenguaje multiparadigma desarrollado en el Laboratorio de Sistemas de Programaci\'on de la Universidad Cat\'olica de Louvain con prop\'ositos educativos.
El lenguaje fue inicialmente dise\~nado por Gert Smolka y sus estudiantes en 1991. Para 1996 el desarrollo de Oz continuaba en cooperaci\'on con el grupo de investigaci\'on de Seif Haridi y Peter Van Roy en el Instituto Sueco de Ciencias de la Computaci\'on. Desde 1999, Oz ha sido continuamente desarrollado por el grupo internacional conocido como \emph{el Consorcio Mozart}, el cual consis\'ia originalmente de la Universidad de Saarland, Universidad Cat\'olica de Louvain y el Instituto Sueco de Ciencias de la Computaci\'on. En 2005, la responsabilidad del desarrollo y mantenimiento de Oz se transfiri\'o a los miembros principales del grupo, \emph{la Junta de Mozart}, con el expreso prop\'osito de abrir el desarrollo de Mozart a una comunidad m\'as grande.
El sistema de programaci\'on Mozart es la implementaci\'on principal de Oz. Ha sido publicado con una liscensia \emph{open source} por el Consorcio Mozart. Tambi\'en ha sido llevado a diferentes plataformas de Unix, FreeBSD, Linux, Windows y OS X.

\section{Importancia y usos}
Oz fue construido principalmente con prop\'ositos educativos ya que contiene la mayor\'ia de los paradigmas de programaci\'on incluyendo l\'ogico, funcional, imperativo, orientado a objetos, programaci\'on concurrente, restrictivo, distribuido, entre otros.



\section{Tipos de datos}
Oz es un lenguaje con muy pocos tipos de datos nativos:

\begin{itemize}
\item N\'umeros: punto flotante y enteros.
\item Records: para agrupar data.
\item Tuplas: records con caracter\'isticas enteras en orden ascendente.
\item Listas: estructuras lineares simples.
\end{itemize}

\section{Expresiones}
\begin{table}[H]
\centering
\caption{Operadores}
\label{my-label}
\begin{tabular}{ll}
Operador                                               & Categor\'ia                                                           \\
Nada                                                   & Nueva l\'inea                                                         \\
/* ... */                                              & Comentarios                                                           \\
\%                                                     & Comentarios                                                           \\
\=, =\textless, \textless, \textgreater=, \textgreater & Comparaciones                                                         \\
Pow                                                    & Potencia                                                              \\
Log                                                    & Logaritmo                                                             \\
mod                                                    & Divisi\'on modular                                                    \\
Sqrt / Exp / Abs                                       & Ra\'iz cuadrada / Funci\'on exponencial \textbackslash Valor Absoluto \\
Sin / Cos / Tan                                        & B\'asicas trigonom\'etricas                                           \\
Asin / Acos / Atan                                     & Inversas trigonom\'etricas                                           
\end{tabular}
\end{table}

\section{Estructuras de control}
\begin{table}[H]
\centering
\caption{Estructuras de control}
\label{my-label}
\begin{tabular}{ll}
Instrucci\'on                              & Uso                 \\
try a catch exn then ... end               & Atrapar excepciones \\
raise ... end                              & Arrojar excepciones \\
if c then ... end                          & if\_then            \\
if c then b1 elseif c2 then b2 else b3 end & if\_then\_else     
\end{tabular}
\end{table}
\section{Caracter\'isticas principales}
Oz tiene sem\'antica formal simple y una implementaci\'on eficiente. Tambi\'en es un lenguaje que facilita la concurrencia, similar a Erlang.

\section{Caracter\'isticas distintivas}
Adem\'as de la programaci\'on multiparadigma, las mayores fortalezas de Oz est\'an en la programaci\'on restrictiva y la programaci\'on distribuida. Debido a su dise\~no, Oz es capaz de implementar un modelo de programaci\'on de red transparente. Este modelo hace f\'acil programar aplicaciones abiertas y tolerantes de fallos con el lenguaje. Para la programaci\'on restrictiva, Oz introduce la idea de \emph{espacios de computaci\'on}, dentro de los cuales se permite la b\'usqueda definida por el usuario y la distribuci\'on de estrategias ortogonal al dominio de las restricciones.
\section{Ventajas y desventajas}
\subsection{Ventajas}
\begin{itemize}
\item Dada su capacidad para la programci\'on multiparadigma, resulta muy \'util para prop\'ositos educativos.
\end{itemize}
\subsection{Desventajas}
\begin{itemize}
\item La velocidad de ejecuci\'on de un programa producido por el compilador Mozart es muy lenta (alrededor de 50 veces m\'as lento que el GCC para C).

\end{itemize}
\section{Ejemplo}
Implementaci\'in de la divisi\'on con el algoritmo trial:
\begin{lstlisting}
fun {Sieve Xs}
   case Xs of nil then nil
   [] X|Xr then Ys in
      thread Ys = {Filter Xr fun {$ Y} Y mod X \= 0 end} end
      X|{Sieve Ys}
   end
end
\end{lstlisting}

\section{Referencias}

\begin{thebibliography}{99}

\bibitem{wiki} Oz (programming language) (2016). . In \emph{Wikipedia}. Retrieved from  \texttt{https://en.wikipedia.org/wiki/Oz\_(programming\_language)}

\bibitem{oz} Syntax in oz. Retrieved September 20, 2016, from \texttt{http://rigaux.org/language-study/syntax-across-languages-per-language/Oz.html}


\end{thebibliography}

\end{document}
