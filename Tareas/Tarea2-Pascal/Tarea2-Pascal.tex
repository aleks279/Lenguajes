\documentclass{IEEEtran}
\usepackage{graphicx}
\usepackage{fancyhdr}
\usepackage{listings}
\lstset
{
    language=Pascal,
    breaklines=true,
    basicstyle=\tt\scriptsize,
    keywordstyle=\color{blue},
    identifierstyle=\color{magenta},
}

\graphicspath{ {images/} }
\pagestyle{fancy}
\fancyhf{}
\rhead{Tarea 2 - Pascal}
\rfoot{P\'agina \thepage}

\begin{document}
\begin{titlepage}
  \centering
  {\scshape\LARGE Instituto Tecnol\'ogico de Costa Rica \par}
  \vspace{1cm}
  {\scshape\Large Tarea 2 - Pascal\par}
  \vspace{1.5cm}
  {\Large\itshape Ariel Herrera\par}
  {\Large\itshape Sa\'ul Zamora\par}
  \vfill
  profesor\par
  M. Sc. Sa\'ul Calder\'on Ram\'irez \textsc{}

  \vfill

% Bottom of the page
  {\large \today\par}
\end{titlepage}

\section{Datos hist\'oricos}
El lenguaje de programaci\'on Pascal est\'a nombrado en honor al matem\'atico franc\'es Blaise Pascal y fue desarrollado por Niklaus Wirth. Antes de su trabajo en Pascal, Wirth desarroll\'o \emph{Euler} y \emph{ALGOL W} y luego trabaj\'o en lenguajes similares a pascal como \emph{Modula-2} y \emph{Oberon}.

Pascal fue el lenguaje de alto nivel mayormente utilizado en el desarrollo de \emph{Apple Lisa} y tambi\'en durante los primeros a\~nos de \emph{Macintosh}.

\section{Importancia y usos}
Inicialmente, Pascal era muy utilizado para la ense\~nanza de la programaci\'on. Generaciones de estudiantes usaron Pascal como un lenguaje introductorio durante la universidad. Variantes de Pascal han sido utilizadas frecuentemente para todo desde proyectos de investigaci\'on orientados a juegos de PC y sistemas embebidos.

Object Pascal es utilizado hoy en d\'ia para el desarrollo de aplicaciones Windows pero tambi\'en tiene la habilidad de compilar el mismo c\'odigo a Mac, iOS y Android. Otra versi\'on multiplataforma llamada \emph{Free Pascal}, junto con el IDE \emph{Lazarus}, es popular entre los usuarios Linux ya que ofrece el tipo de desarrollo de \emph{escribe el c\'odigo una vez y compilalo donde sea}.

\section{Tipos de datos}
Al igual que en otros lenguajes, un tipo en Pascal define un rango de valores que una variable es capaz de almacenar y tambi\'en define que operaciones pueden ser ejecutadas sobre dicha variable.
\begin{itemize}
\item \emph{integer}: n\'umeros enteros
\item \emph{real}: n\'umeros de punto flotante
\item \emph{boolean}: valores de falso o verdadero
\item \emph{char}: un caracter de un set ordenado de caract\'eres
\item \emph{string}: grupo o hilera de caract\'eres
\item \emph{subrango de tipos}: se pueden contruir subrangos de cualquier tipo de dato ordinal
\begin{lstlisting}
var
  x : 1..10;
  y : 'a'..'z';
\end{lstlisting}
\item \emph{sets}: en contraste con otros lenguajes de tu tiempo, Pascal soporta tipos de datos set
\begin{lstlisting}
var
  Set1 : set of 1..10;
  Set2 : set of 'a'..'z';
\end{lstlisting}
\item \emph{declaraciones de tipos}: los tipos pueden ser definidos usando otros tipos por medio de declaraciones
\begin{lstlisting}
type
  x = integer;
  y = x;
\end{lstlisting}
O declaraciones m\'as complejas:
\begin{lstlisting}
type
  a = array[1..10] of integer;
  b = record
        x : integer;
        y : char
      end;
  c = file of a;
\end{lstlisting}
\item \emph{tipo archivo}: como se mostro en el c\'odigo anterior, en Pascal, los archivos son secuencias de componentes
\item \emph{punteros}: Pascal soporta el uso de punteros
\begin{lstlisting}
type
  pNode = ^Node;
  Node  = record
          a : integer;
          b : char;
          c : pNode  {extra semicolon not strictly required}
          end;
var
  NodePtr : pNode;
  IntPtr  : ^integer;
\end{lstlisting}
\end{itemize}
\section{Expresiones}
Las expresiones en Pascal se dan en asignaciones o pruebas. Las expresiones producen un valor de un cierto tipo. Son construidas con dos componentes: los operadores y los operandos. Usualemente los operadores son binarios; requieren dos operandos. Tambi\'en hay operadores unarios, que requieren un \'unico operando.
\begin{table}[]
\centering
\caption{My caption}
\label{my-label}
\begin{tabular}{lll}
Operador                                                                      & Precedencia & Categor\'ia             \\
Not, @                                                                        & M\'as alta  & Operadores unarios      \\
* / div mod and shl shr as \textless\textless \textgreater\textgreater        & Segunda     & M\'ultiples operadores  \\
+ - or xor                                                                    & Tercera     & Operadores de adici\'on \\
= \textless\textgreater \textless \textgreater \textless= \textgreater= in is & M\'as baja  & Operadores relacionales
\end{tabular}
\end{table}
\section{Almacenamiento}
\section{Estructuras de control}
Pascal es un lenguaje estructurado, lo que significa que el control de flujo est\'a estructurado en estatutos est\'andares, usualmente sin el comando \emph{goto}.
\begin{lstlisting}
while a <> b do  WriteLn('Waiting');

if a > b then WriteLn('Condition met')   {no semicolon allowed!}
           else WriteLn('Condition not met');

for i := 1 to 10 do  {no semicolon for single statements allowed!}
  WriteLn('Iteration: ', i);

repeat
  a := a + 1
until a = 10;

case i of
  0 : Write('zero');
  1 : Write('one');
  2 : Write('two');
  3,4,5,6,7,8,9,10: Write('?')
end;
\end{lstlisting}
\section{Caracter\'isticas principales}
\section{Caracter\'isticas distintivas}
\section{Ventajas y desventajas}
\section{Ejemplo}

\section{Referencias}

\begin{thebibliography}{99}

\bibitem{wiki} Pascal (programming language) (2016). . In \emph{Wikipedia}. Retrieved from \texttt{https://en.wikipedia.org/wiki/Pascal_(programming_language)}

\end{thebibliography}

\end{document}
