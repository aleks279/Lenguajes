\documentclass[11pt]{article}
\title{\textbf{Tarea 1 - \LaTeX{}}}
\author{Saul Zamora Castro\\200835773\\}
\date{}
\begin{document}

\maketitle

\section{Datos historicos}
\LaTeX{} es un sistema de composicion de texto, orientado a la creacion de documentos que presentan una alta calidad tipografica. Dadas sus caracteristicas y posibilidades, es usado particularmente en la generacion de articulos y libros cientificos que incluyen entre otas cosas, expresiones matematicas.

El sistema esta conformado por un gran conjunto de macros de TeX, escrito por Leslie Lamport en 1984 con la intencion de facilitar la composicion tipografica de TeX, que fue creado por Donald Knuth.

El ser de codigo abierto permite que muchos de sus usuarios realicen nuevas utilidades que extiendan las capacidades del sistema; las cuales no siempre tienen la misma intencion con la que \LaTeX{} fue creado. Para solucionar este problema, en 1989 Lamport y otros desarrolladores iniciaron lo que se conoce como el `Proyecto LaTeX3'. En 1993 se anuncio una reestandarizacion completa de \LaTeX{} mediante una nueva versio que incluyera la mayor parte de las extensiones adicionales para dar uniformmidad al conjunto y evitar la fragmentacion entre versiones incompatibles.

Una nueva version del sistema sale al publico cada año, aunque las diferencias entre una y otra suelen ser minimas, siempre estan bien documentadas.

\section{Importancia y usos academicos}
\LaTeX{} trabaja con una filosofia diferente de la de los procesadores de texto habituales (conocida como WYSIWYG `What You See Is What You Get' que significa `lo que ves es lo que obtienes'). Pero a diferencia de los otros procesadores, con \LaTeX{} el escritor puede dedicarse exclusivamente al contenido sin tener que preocuparse por los detalles de formato. Ademas de sus capacidades graficas para representar expresiones matematicas y formulas complicadas, notacion cientifica y musical, permite estructurar facilmente el documento, lo cual ofrece comodidad y lo hace util para articulos academicos y libros tecnicos.

La facilidad que presenta \LaTeX{} para insertar imagenes y graficas son la necesidad de estar pendiente de su ubicacion final en el documento (cosa que si hay que tener presente en los procesadores de texto mas habituales como Microsoft Word) es uno de los aspectos que lo hacen preferido por estudiantes para documentos como tesis.

Los documentos generados por \LaTeX{} son de muy alta calidad, especialmente cuando hay formulas matematicas involucradas. Otro factor que hace ventajoso a \LaTeX{} sobre otros procesadores de texto, es que no depende de ninguna plataforma para funcionar; asi como puede ser usado en Windows, puede ser utilizado en Linux o Mac OS.

\section{Estilos importantes}


\section{Uso}

\begin{thebibliography}{99}
- TeXnicCenter, About \LaTeX{} - `http://www.texniccenter.org/about/about-\LaTeX{}'
\\- \LaTeX{}, Wikipedia - `https://es.wikipedia.org/wiki/\LaTeX{}'
\end{thebibliography}
\end{document}